\documentclass[10pt, a4paper]{article}
\usepackage[T1,T2A]{fontenc}
\usepackage[utf8]{inputenc}
\usepackage[english,russian]{babel}
\setlength{\parskip}{1ex}

\title{Первое дз}
\author{Vadim Bayduyk}
\date{\today}

\begin{document}
	{\fontfamily{LinuxLibertineT-TLF}\selectfont
		\noindent\LARGE\textbf{Частотный анализ}\\\\
		\normalsize
		\begin{tabular}{ll}
			\indent Имя входного файла: & standart input\\
			\indent Имя выходного файла: & standart output\\
		\end{tabular}
		\hyphenation{ ощу-тимый сравне-ние значе-ний значе-ние обособ-лены}
		
		\noindent В криптографии в алгоритмах расшифровки текстов, которые используют ощутимый перебор, для автоматической проверки того, что расшифрованный шифротекст потенциально может быть осмысленным сообщением используют частотный анализ. Один из самых примитивных частотных анализов  - сравнение среднего отклонения частоты встречаемости символов от реальных значений с эталонным значением. Будем считать, что текст прошел частотный анализ, если полученное среднее отклонение не превосходит эталонное значение. Ваша задача  провести частотный анализ заданного текста.
		
		\noindent\large\textbf{Формат входного файла}\\
		\normalsize В первой строчке вводится целое положительное число $n$ и положительное(не обязательно целое) число $A$, количество слов в тексте и эталонное значение среднего отклонения от реальных значений частоты встречаемости символов соответтвенно, далее вводится текст из $n$ слов возможно со знаками препинания. Гарантируется, что все знаки препинания(кроме дефиса) не могут быть обособлены пробелами с обеих сторон, т.е. не возможен текст <<Alina learns how to write programs  , let's wish her good luck !>>. Так же гарантируется, что все слова текста на английском языке. 
		
		\noindent\large\textbf{Формат выходного файла}\\\normalsize Выведите <<YES>>, если текст проходит частотный анализ или <<NO>> иначе.
		
		\noindent\large\textbf{Пример}\normalsize
		
		\noindent\begin{tabular}{|p{5,5cm}|p{5,5cm}|}
			\hline
			\multicolumn{1}{|c|}{standart input} & \multicolumn{1}{c|}{standart output}\\\hline
			11 3 & YES \\
			Alina learns how to write programs, & \\
			let's wish her good luck! &\\\hline
		\end{tabular}
		
		\noindent Частота встречаемости символов английского языка представлена в таблице
		
		
		\newpage
		\begin{tabular}{|c|c|}
			\hline
			Буква & Частота \\ \hline
			E & 12,70 \\ \hline
			T & 9,06 \\ \hline
			A & 8,17 \\ \hline
			O & 7,51 \\ \hline
			I & 6,97 \\ \hline
			N & 6,75 \\ \hline
			S & 6,33 \\ \hline
			H & 6,09 \\ \hline
			R & 5,99 \\ \hline
			D & 4,25 \\ \hline
			L & 4,03 \\ \hline
			C & 2,78 \\ \hline
			U & 2,76 \\ \hline
			M & 2,41 \\ \hline
			W & 2,36 \\ \hline
			F & 2,23 \\ \hline
			G & 2,02 \\ \hline
			Y & 1,97 \\ \hline
			P & 1,93 \\ \hline
			B & 1,49 \\ \hline
			V & 0,98 \\ \hline
			K & 0,77 \\ \hline
			X & 0,15 \\ \hline
			J & 0,15 \\ \hline
			Q & 0,10 \\ \hline
			Z & 0,05 \\ \hline
		\end{tabular}
	}
	
\end{document}
