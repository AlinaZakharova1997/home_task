\documentclass[10pt, a4paper]{article}
\usepackage[T1,T2A]{fontenc}
\usepackage[utf8]{inputenc}
\usepackage[english,russian]{babel}
\setlength{\parskip}{1ex}

\begin{document}
	{\fontfamily{LinuxLibertineT-TLF}\selectfont
		\noindent\LARGE\textbf{Интересная игра}\\\\
		\normalsize
		\begin{tabular}{ll}
			\indent Имя входного файла: & standart input\\
			\indent Имя выходного файла: & standart output\\
		\end{tabular}
		
		\noindent Вася и Петя играют в игру. Вася записал на доске много чисел(не обязательно уникальных), запомнил их всех и отвернулся от нее. Петя в любой момент может поменять ВСЕ одинаковые числа на другие тоже одинаковые числа. Так, например, если на доске были записаны числа 1, 3, 5, 3, 2, 1, 1, 5, и если Петя решил заменить все 3 на 7, то на доске окажутся числа 1, 7, 5, 7, 2, 1, 1, 5. В любой момент Петя может попросить Васю посчитать количество повторов некоторого числа. Если Вася отвечает неправильно, он проиграл. Ваша задача не дать Васе проиграть.
		
		\noindent\large\textbf{Формат входного файла}\\
		\normalsize В первой строчке вводятся целые положительные числа n и m, количество чисел записанных Васей на доске, и количество ходов в игре. В следующей строке вводится n чисел. В следующих m строках вводится описание ходов в формате 
		\begin{enumerate}
				\item 1 a - узнать количество чисел, равных a
				\item 2 a b - заменить все числа a на числа b
		\end{enumerate}
		
		\noindent\large\textbf{Формат выходного файла}\\\normalsize На каждый запрос первого типа выведите ответ в отдельной строке. 
		
		\noindent\large\textbf{Пример}\normalsize
		
		\noindent\begin{tabular}{|p{5,5cm}|p{5,5cm}|}
			\hline
			\multicolumn{1}{|c|}{standart input} & \multicolumn{1}{c|}{standart output}\\\hline
			8 5 & 3 \\
			1 3 5 3 2 1 1 5 & 5 \\
			1 1 & 0\\
			2 3 7 & \\
			2 1 7 & \\
			1 7 & \\
			1 3 & \\
			& \\\hline
		\end{tabular}
		
	}
	
\end{document}
